\documentclass[a4paper,10pt]{article}\usepackage[utf8]{inputenc}\usepackage{geometry}\geometry{a4paper,total={170mm,257mm},left=20mm,top=20mm,}\begin{document}\title{\vspace{-4cm}Curriculum Vitae}\author{Ryan Kim}\date{}\maketitle\noindentLinkedIn: \href{https://www.linkedin.com/ryankima}{linkedin.com/ryankima} \\GitHub: \href{https://www.github.com/ryankima}{github.com/ryankima} \\Email: ryankima@umich.edu \\Phone Number: 734-787-8832\section*{Education}\noindent\textbf{Michigan State University, Bachelor of Bartending, Minor in being unpleasant} \hfill Aug 2021 - Expected May 2025\\Location: East Lansing, MI \\GPA: 2.7 \\Relevant Coursework: Beverage Management, Principles of Distilling, Customer Relations, Sales Strategies, Event Planning, Conflict Resolution\noindent\textbf{Technische Universit�t Berlin, Study Abroad,} \hfill May 2022 - June 2022\\Location: Berlin, Germany \\Specialization: International laboratory experience in robotics programming \section*{Experience}\noindent\textbf{Engineering Development Group Intern} \hfill May 2023 - Aug 2023 \\\textit{The Mathworks}, Natick, MA \\\begin{itemize}\item Pioneered a code generation pipeline using Vulkan and IREE technologies to provide 2x speed improvement for matrix operations through GPU acceleration; the innovation also allowed for the utilization of pretrained machine learning models.\item Optimized future development strategies by exploiting open source software; this had a two-fold beneficial effect - reduced development time and introduced unique code optimization processes in the pipeline.\end{itemize}\noindent\textbf{Student Fellow} \hfill June 2022 - May 2023 \\\textit{Consortium for Monitoring, Technology, and Verification - University of Michigan}, Ann Arbor, MI \\\begin{itemize}\item Engineering optimization endeavours lead to an overhaul of existing software architecture for low-cost Geiger Counter; this significantly enriched hardware performance, greater device compatibility, and lower maintenance requirements.\item Delivered analytical presentations at national conferences presenting research findings and future scope, effectively enhancing public understanding and interest in our research.\end{itemize}\noindent\textbf{Instructional Assistant} \hfill Aug 2022 - Dec 2022 \\\textit{University of Michigan}, Ann Arbor, MI \\\begin{itemize}\item Developed course curricula focused on concepts of electrical engineering, radiation science, and radiation detection; this effectively imbibed a team-centric learning environment leading to higher student engagement and learning outcomes.\item Championed safety compliance protocols, personal protective equipment, and execution methodologies to ensure student safety during practical labs.\end{itemize}\noindent\textbf{Assistant in Research} \hfill Sep 2021 - May 2022 \\\textit{University of Michigan}, Ann Arbor, MI \\\begin{itemize}\item Developed and tested firmware for custom sensors linked to autonomous drones running on PX4 autopilot ensuring stable and reliable communication through SPI and UART.\item Executed computational strength-power usage trade-off analysis for system computers in aerial drone applications, leading to an optimal design.\end{itemize}\section*{Languages}C/C++, Python, Java, HTML5, CSS, JavaScript, C#, SQL, Matlab\section*{Tools}XCode, Visual Studio, Git, Github, Computer-aided Design (CAD), Raspberry Pi, SolidWorks, MySQL, MongoDB, Android Studio, Flask, IREE, MLIR, Vulkan, GPU, Compilers\end{document}